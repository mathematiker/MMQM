\documentclass{mycourse}
\usepackage[all]{xy}
\usepackage{array}
%\usepackage[a4paper,left=1 cm,right= 1cm, top=1cm, bottom=2cm] {geometry}

% Für Text zwischen enumerate-items aber ohne Bezug zu einem bestimmten item.
\makeatletter
\newcommand{\interitemtext}[1]{%
\begin{list}{}
{\itemindent=0mm\labelsep=0mm
\labelwidth=0mm\leftmargin=0mm
\addtolength{\leftmargin}{-8mm}}
\item #1
\end{list}}
\makeatother

\ifthenelse{\isundefined{\psmallmatrix}}{
	\newenvironment{psmallmatrix}{\left(\begin{smallmatrix}}{\end{smallmatrix}\right)}
}{}

\newcommand{\dist}{\operatorname{dist}}
\newcommand{\Ran}{\operatorname{Ran}}
\renewcommand{\eqref}[1]{\hyperref[#1]{(#1)}}

\title{Mathematische Methoden der Quantenmechanik}
\author{}
\date{\today}

\begin{document}

\maketitle
\tableofcontents

\chapter{Lineare Operatoren}
\section{Spektrum linearer Operatoren}
Sei $H$ ein Hilbertraum über $\C$ mit Skalarprodukt $\langle \cdot, \cdot \rangle$. Sei $D\subset H$ ein linearer Teilraum von $H$ und $A:D\to H$ ein linearer Operator. $D=D(A)$ mit \emph{Definitionsbereich} von $A$ und $AD=\{A\phi|\, \phi\in D\}$ heißt \emph{Wertebereich} (range) von $A$.

Die \emph{Resolventenmenge} $\rho(A)$ von $A$ ist definiert als $\rho(A)=\{z\in \C|\, z-A: D\to H \text{ bijektiv $(z-A)^{-1}$ ist beschränkt}\}.$

Die \emph{Resolvente} von $A$ ist die Abbildung $R=R_A: \rho(A) \to \mathcal L(H)$, gegeben durch $R(z)=(z-A)^{-1}$.

Das \emph{Spektrum} $\sigma(A)$ von $A$ ist definiert durch $\sigma(A)=\C\setminus \rho(A)$. Man zerlegt $\sigma(A)$ in
\begin{seg}[Punktsprektum]
Das \emph{Punktspektrum} ist gegeben durch
\[
\sigma_p(A)=\{z\in \C |\, (z-A): D\to H \text{ nicht injektiv}\}= \text{Menge der Eigenwerte.}
\]
\end{seg}
\begin{seg}[stetiges Spektrum]
Das \emph{kontinuierliche Spektrum} ist gegeben durch
\[
\sigma_c(A)=\{z\in \C|\, (z-A): D\to H \text{ injektiv, nicht surjektiv aber } \overline{(z-A)D}= H\}
\]
\end{seg}
\begin{seg}[Resdual-Spektrum]
Das \emph{Residual-Spektrum} oder auch \emph{Rest-Spektrum} ist gegeben durch
\[
\sigma_r(A)=\{z\in \C|\, (z-A):D \to H \text{ injektiv, aber } \overline{(z-A)D}\neq H\}.
\]
\end{seg}
\begin{nt*}
\begin{enumerate}[1)]
\item Falls $A$ abgeschlossen, d.h. wenn $\Gamma_A= \{(\phi, A\phi)|\, \phi \in D\}$ abgeschlossen ist in $H\times H$, und $z-A: D \to H$ bijektiv ist, dann ist $z-A$ und somit auch $(z-A)^{-1}$ abgeschlossen. Nach dem Graphensatz ist somit $(z-A)^{-1}$ beschränkt. D.h. wenn $A$ abgeschlossen ist, dann $\rho(A)=\{z\in \C|\, z-A: D \to H \text{ bijektiv}\}$. Umgekehrt wenn $\rho(A) \neq \emptyset$ dann existiert $z\in \C$, so dass $(z-A):D\to H$ bijektiv und $(z-A)^{-1}$ beschränkt und somit abgeschlossen ist, dann
\[
\rho(A)=\{z\in \C|\, z-A: D\to H \text{ bijektiv}\}.
\]
Umgekehrt wenn $\rho(A)\neq \emptyset$, dann existiert $z\in \C$, so dass $(z-A):D\to H$ bijektiv und $(z-A)^{-1}$ beschränkt und somit abgeschlossen ist. Es folgt, dass $(z-A)$ und daher auch $A$ abgeschlossen ist.
\item wenn $A$ abgeschlossen ist, dann $\sigma(A)=\sigma_p(A) \dot\cup \sigma_c(A) \dot\cup \sigma_r(A)$.
\end{enumerate}
\end{nt*}
\begin{ex*}
\begin{enumerate}[1)]
\item Sei $H= l^2(\N_0)=\{(x_n)_{n=0}^\infty |\, x_n \in \C,\, \sum_{n=0}^\infty|x_n|^2<\infty\}$ und sei $A: H\to H$ gegeben durch $A(x_0, x_1, x_2,...)=(x_1, x_2,...)$. Dann ist $A\in \mathcal L(H)$ mit $\|A\|=1$
\[
\sigma_p(A)=\{z\in \C|\, |z|<1\}, \quad \sigma_c(A)=\{z\in \C|\, |z|=1\}, \quad \sigma_r(A)=\emptyset.
\]
Der Beweis ist eine leichte Übung.
\item $H=L^2(\R)$, $D=\{\phi\in L^2(\R)| \phi'\in L^2\}$ und $A\phi=-i \phi'$. Dann $\sigma(A)=\sigma_c(A)=\R$. Entsprechend $\sigma_p(A)=\sigma_r(A)= \emptyset$. Beweis ist eine leichte Übung.
\end{enumerate}
\end{ex*}
\begin{st}\label{1.1}
Sei $A: D \subset H\to H$ ein lin. Operator mit Resolvente $R$. Dann gilt
\begin{enumerate}[a)]
\item $R(z)-R(w)=(w-z) R(z) R(w)$,
\item $R(z) R(w)=R(w) R(z)$,
\item $R(z) A\subset A R(z)=zR(z)-I$
\end{enumerate}
für alle $z,w\in \rho(A)$.
\end{st}
\begin{proof}
siehe Satz 1.1 aus dem Skript.
\end{proof}
\begin{st} \label{1.2} Sei $A$ ein linearer Operator in $H$. Dann ist $\rho(A)$ offen, $\sigma(A)$ abgeschlossen und $R: \rho(A) \to \mathcal L(H)$ ist analytisch genauer: ist $z_0\in \rho(A)$, dann $B(z_0, \| R(z_0)\|^{-1}) \subset \rho(A)$ und in dieser Kreisscheibe gilt
\[
R(z)=\sum_{n=0}^\infty (-1)^n R(z_0)^{n+1} (z-z_0)^n.
\]
Es gilt außerdem $\|R(z_0)\|\ge \frac{1}{\dist(z_0, \sigma(A))}$.
\end{st}
\begin{proof}
siehe Satz 1.2 aus dem Skript.
\end{proof}
\begin{seg}[Erinnerung]
Sind $A,B$ lineare Operatoren in $H$, dann heißt $B$ \emph{Erweiterung} von $A$ falls $D(A) \subset D(B)$ und $A\phi=B\phi$ für alle $\phi \in D(A)$. Der Operator $A$ \emph{vertauscht} mit $L\in \mathcal L(H)$, falls $LA\subset AL$, d.h. $LD(A)\subset D(A)$ und für alle $\phi \in D(A)$ gilt $LA\phi=AL\phi$.
\end{seg}
\begin{st}
Sei $A$ ein linearer Operator in $H$, $z_0\in \rho(A)$ und $L\in \mathcal L(H)$. Dann gilt $LR(z_0)=R(z_0)L \implies LA \subset AL \implies LR(z)=R(z)L$ für alle $z\in \rho(A)$.
\end{st}
\begin{proof}
siehe Satz 1.4 aus dem Skript.
\end{proof}

Sei $P\in \mathcal L(H)$ ein Projektor (d.h. $P^2=P$), dann ist auch $(I-P)$ ein Projektor und $H=M \oplus N$, wobei $M=PH=\ker(I-P)$ und $N=(1-P)H=\ker(P)$. D.h. jeder Vektor $\phi\in H$ lässt sich eindeutig zerlegen in $\phi=\phi_1+\phi_2$ mit $\phi_1\in M$ und $\phi_2\in N$. Dabei gilt $\phi_1=P\phi$ und $\phi_2=(1-P)\phi$.

Ist $A$ ein linearer Operator in $H$ welcher mit den Projektoren $P$ vertauscht, also $PA\subset AP$. Dann wird $A$ durch $M=PH, N=(I-P)H$ zerlegt in die beiden Operatoren
\[
A_p:=A \upharpoonright PH: PD(A) \to PH, \quad A \upharpoonright \bar PH: \bar P D(A) \to \bar{P}H
\] 
wobei $\bar P:=I-P$.
\begin{lem}\label{1.4}
Ist $A$ ein abgeschlossener linearer Operator in $H$ und $p\in \mathcal L(H)$ ein Projektion mit $PA \subset AP$. Dann gilt $\sigma(A)=\sigma(A_p)\dot\cup \sigma(A_{\bar{P}})$.
\end{lem}
\begin{proof}
Sei $z\in \C$. Dann ist $(z-A):D(A) \to H$ genau dann bijektiv, wenn $z-A_P: PD(A)\to PH$ und $z-A_{\bar{P}}:\bar P D(A) \to \bar P H$ bijektiv sind. Also $\rho(A)=\rho(A_p) \cap \rho(A_{\bar P})$ und somit $\sigma(A)=\sigma(A_P) \cup \sigma(A_{\bar P})$. Tatsächlich gilt
\[
(z-A_P)^{-1}=(z-A)^{-1} \upharpoonright PH, \quad (z-A_{\bar P})^{-1} = (z-A)^{-1} \upharpoonright \bar P.
\]
\end{proof}
\section{Rieszprojektoren und Laurentreihe der Resolvente}
\subsection{Der Riesz-Projektor}
Sei $A$ ein abgeschlossener linearer Operator in $H$ und sei $\lambda\in \sigma(A)$ ein isolierter Punkt des Spektrums von $A$. D.h. es gibt ein $\eps>0$ so, dass $\overline{B_\eps(\lambda)} \cap \sigma(A)=\{\lambda\}$. Wir definieren den \emph{Riesz-Projektor} $P$ durch
\[
P:= \frac{1}{2\pi i} \oint_{|z-\lambda|=\eps} R(z) \, \dx[z],
\]
wobei das Integral als Riemannintegral aufgefasst ist.  
\begin{st}\label{1.5}
\begin{enumerate}[a)]
\item $P$ ist ein Projektor.
\item $PA\subset AP=\frac{1}{2\phi} \oint zR(z)\, \dx[z] \in \mathcal L(H)$.
\item $\sigma(A_P)=\{\lambda\}$,\quad $\sigma(A_{\bar P})=\sigma(A) \setminus \{\lambda\}$.
\end{enumerate}
\end{st}
\begin{proof}
Siehe Satz 1.6 des Skripts. Man merke an, dass $\sigma(A)=\{\lambda\} \dot\cup (\sigma(A)\setminus\{\lambda\})$ in zwei abgeschlossene Teile zerfällt.
\end{proof}
\begin{nt*}
Satz \ref{1.5} lässt sich verallgemeinern. Falls $\sigma(A)$ in zwei disjunkte abgeschlossene Teile $\sigma_1$ und $\sigma_2$ zerfällt. D.h. $\sigma(A)=\sigma_1\cup \sigma_2$ mit $\sigma_1\cap \sigma_2=\emptyset$ und $\sigma_1, \sigma_2$ abgeschlossen, wobei $\sigma_1$ kompakt ist und $P\in \mathcal L(H)$ definiert wird durch
\[
P=\frac{1}{2\pi i} \oint_\gamma (z-A)^{-1} \, \dx[z],
\]
wobei die Windungszahl $n(\gamma, z)=1$ für $z\in \sigma_1$ und $n(\gamma,z)=0$ für $z\in \sigma_2$ und $\tr(\gamma)\subset \rho(A)$. Dann gilt
\begin{st}\label{1.6}
\begin{enumerate}[a)]
\item $P$ ist ein Projektor.
\item $PA\subset AP=\frac{1}{2\phi} \oint zR(z)\, \dx[z] \in \mathcal L(H)$.
\item $\sigma(A_P)=\sigma_1$,\quad $\sigma(A_{\bar P})=\sigma_2$.
\end{enumerate}
\end{st}
\end{nt*}
\subsection{Laurentreihe der Resolventen}
Nach Satz \ref{1.5} gilt für $z\in B_\eps(\lambda)\setminus\{\lambda\}$
\[
(z-A)^{-1}=(z-A)^{-1}P+(z-A)^{-1}\bar P=(z-A_P)^{-1} P + (z- A_{\bar P})^{-1} \bar P,
\]
wobei $A_P$ und $A_{\bar P}$ die Einschränkungen von $A$ auf $PH$ bzw. $PH$ sind. Ausserdem ist $\sigma(A_p)=\{\lambda\}$, $\sigma(A_{\bar P}) \subset \sigma(A) \setminus\{\lambda\}$. Also gilt
\[
(z-A_P)^{-1} = \sum_{n\ge 0} \frac{(A_P-\lambda)^n}{(z-\lambda)^{n+1}}, \quad (z-A_{\bar P})^{-1} \stackrel{\ref{1.2}}=(-1)^n (\lambda-A^n)^{-n-1} (z-\lambda)^n
\]
für alle $z\in B_\eps(\lambda)\setminus\{\lambda\}$. Hierbei sei anzumerken, dass die Laurentreihenentwicklung konvergiert, da der Spektralradius von $(A_P-\lambda)$ gerade $0$ ist. (Näheres siehe Appendix)
\begin{st}
Für $0<|z-\lambda|<\eps$ gilt
\[
R(z)=\sum_{n=1}^\infty D^n(z-\lambda)^{-n-1} + \frac{P}{z-\lambda} + S(z), \quad S(z)=\sum_{n=0}^\infty (-1)^n S(\lambda)^n (z-\lambda)^n,
\]
wobei $D:=(A_P-\lambda)P=(A-\lambda)P$ und $S(z):= (z-A_{\bar P})^{-1}\bar P$ die reduzierte Resolvente ist.
\end{st}
\begin{nt*}
\begin{enumerate}[1)]
\item Da der Hauptteil von $R(z)$, d.h.
\[
\sum_{n=1}^\infty D^n(z-\lambda)^{-n-1} + \frac{D}{z-\lambda},
\]
konvergent ist für alle $z\neq \lambda$, gilt $\limsup_{n\to \infty} \|D^n\|^{1/n}=0$. Man sagt: $D$ sei \emph{quasi-nilpotent}.
\item Falls $D$ nilpotent ist, d.h. $D^m=0$ für ein $m\in N$ und $D^{m-1}\neq 0$. Dann ist $\lambda$ ein Eigenwert von $A$. 
\begin{proof}
Wähle $\phi \neq 0$ mit $D^{m-1}\phi\neq0$. Dann ist $D(D^{m-1}\phi)=0$, d.h.
\[
(A-\lambda) (D^{m-1} \phi)=(A-\lambda) P(D^{m-1} \phi)= D^m \phi =0
\]
\end{proof}
\item Falls $m:= \dim(PH)<\infty$, dann ist $D^m=0$ und somit $\lambda$ ein Eigenwert von $A$. (Übung)

Dann heißt $\lambda$ ein \emph{diskreter Eigenwert} von $A$. Das Komplement der diskreten Eigenwerte innerhalb $\sigma(A)$ heißt \emph{wesentliches Spektrum} von $A$.
\end{enumerate}
\end{nt*}
\section{Der adjungierte Operator}
Sei $A:D\subset H\to H$ ein \emph{dicht definierter} linearer Operator, d.h. $\overline{D}=H$ (z.B. $D=H$). Der zu $A$ \emph{adjungierte Operator} $A^*: D(A^*)\subset H \to H$ ist wie folgt definiert. 

Wir sagen $\phi \in D(A^*)$ wenn $\phi^*\in H$ existiert, so dass $\langle \phi^*, \eta\rangle = \langle \phi, A\eta\rangle$ für alle $\eta \in D(A)$. Dann ist $\phi^*$ durch $\phi$ eindeutig bestimmt (da $\overline{D(A)}=H$) und man definiert $A^* \phi:= \phi^*$. Die Abbildung $\phi \mapsto A^*\phi$ ist linear. (Übung)

Nach Frechet-Riesz (genaueres siehe Übung) gilt
\[
D(A^*)=\{\phi\in H|\, \eta \mapsto \langle \phi, A\eta\rangle \text{ ist stetig auf } D(A)\}
\]

\begin{st}
\begin{enumerate}[a)]
\item $A\subset B \implies A^*\supset B^*$.
\item $A^*$ ist abgeschlossen.
\item $\ker A^*=(R-A)^\orth$. 
\end{enumerate}
\end{st}
\begin{proof}
a) wurde in der Übung gezeigt. b) wird in Satz 2.1 aus dem Skript bewiesen. Und c) wird in Satz 2.2 aus dem Skript gezeigt.
\end{proof}
\begin{st}
Ist A \emph{abgeschlossen}, dann ist $A^*$ dicht definiert und $(A^*)^*=A$.
\end{st}
\begin{proof}
Es ist
\begin{align*}
(\phi, \phi^*)\in \Gamma(A^*) &\iff \langle \phi^*, \eta\rangle = \langle \phi, A\eta\rangle, \, \forall \eta\in D(A)\\ &\iff D=\langle \phi, A\eta\rangle - \langle \phi^*, \eta\rangle \\ &\iff (\phi, \phi^*) \orth (A\eta, - \eta)
\end{align*}
bezüglich des Skalarprodukts $\langle (\phi_1, \phi_2), (\psi_1, \psi_2)\rangle=\langle \phi_1, \psi_1\rangle + \langle \phi_2, \psi_2\rangle$ von $H\times H$. Sei $V:H\times H \to H \times H$ definiert durch $V(\phi, \psi)=(\psi, -\phi)$. Dann ist $V^2=1$ und $V$ ist unitär. Für jeden Teilraum $E\subset H\times H$ gilt also $(VE)^\orth = V(E^\orth)$. (Übung)

Somit gilt
\begin{align*}
(\phi, \phi^*) \in \Gamma(A^*) &\iff (\phi, \phi^*) \orth V(\eta, A\eta), \, \forall \eta\in D(A)\\
&\iff (\phi, \phi^*)\in (V\Gamma(A))^\orth.
\end{align*}
D.h. $\Gamma(A^*)=(V\Gamma(A))^\orth=V\Gamma(A)^\orth$. Da $\Gamma(A)$ abgeschlossen ist folgt 
\end{proof}

\chapter{Analytische Störungstheorie}
\setcounter{thm}{13}
\begin{st}
Sei $A\subset A^*$. Dann sind äquivalent:
\begin{enumerate}[a)]
\item $A$ ist wesentlich selbstadjungiert.
\item $\overline{\Ran(z_\pm -A)}=H$ für ein $z_+\in \C_+$ und ein $z_-\in \C_-$.
\item $\ker(z_\pm - A^*)= \{0\}$ für $z_\pm \in \C_\pm$
\end{enumerate}
\end{st}
\begin{proof}
Folgt mit Lemma 12, Theorem 3 und Satz 1.8 c)
\end{proof}

\begin{st}[Nelson]
Sei $U: \R \to \mathcal L(H)$ eine s.s.u.G. mit Erzeuger $A$ und sei $D\subset D(A)$ mit $\overline{D}=H$ und $U(t)D\subset D$ für alle $t\in \R$. Dann gilt $A=\overline{A \upharpoonright D}$. Insbesondere ist $A\upharpoonright D$ wesentlich selbstadjungiert.
\end{st}
\begin{proof}
Siehe Satz 4.3 aus dem Skript. Im Beweis wird Satz 2.8 aus dem Skript verwendet. 
\end{proof}

\section{Translationsgruppe und Impulsoperator}
Sei $H=L^2(\R)$ und sei $U$ die s.s.u.G. definiert durch $(U(t)\phi)(x)=\phi(x-t)$. Sei $D=C_0^\infty(\R)$. Dann $\overline{D}=H, U(t)D \subset D$ und
\[
i \frac{d}{dt} U(t) \phi\mid_{t=0}= -i \phi'(x), \quad \phi \in D.
\]
Mit anderen Worten: Wenn $A$ der Erzeuger von $U$ ist
dann gilt $D\subset D(A)$ und $B:=A\upharpoonright D=-i \frac{d}{dx} \upharpoonright D$. Mit dem Satz von Nelson folgt 
\[
A=-i \frac{d}{dx} \upharpoonright C_0^\infty(\R)
\]
dem \emph{Impulsoperator} in der Quantenmechanik. Zum Beweis  zeigen wir, dass für $\phi \in D$ gilt
\[
\|i \frac{U(t) \phi- \phi}{t} - B\phi \|^2 = \|\frac{U(t) \phi - \phi}{t} + i B\phi \|^2 =\int \big| \underbrace{\frac{\phi(x-t)-\phi}{t} + \phi'(x)}_{\to 0, (t\to 0)}\big|^2 \dx \to 0,
\]
wobei $t\to 0$ mit dem Satz über majorisierte Konvergenz.

\section{Rotationsgruppe und Drehimpuls}
Für jedes $R\in SO(3)$ definieren wir als $U(R): L^2(\R^3) \to L^2(\R^3)$ durch
\[
(U(R)\phi)(x):= \phi(R^{-1}x).
\]
$U(R)$ ist unitär und es gilt  $U(R_1 R_2)=U(R_1) U(R_2)$. D.h. $R\mapsto U(R)$ ist eine Darstellung von $SO(3)$ auf $L^2(\R^3)$. Dei Rotation $R_t\in SO(3)$ um eine feste Achse $e\in \R^3$ mit Winkel $t\in \R$ bilden eine einparametrige (abelsche Untergruppe von $SO(3)$. Es gilt $R_{s+t}=R_s R_t$ und $\frac{d}{dt} R_t x|_{t=0}=e \land x$. 

Sei $U(t)\in \mathcal L(L^2(\R^3))$ definiert durch
\[
(U(t)\phi)(x)=\phi(R_t^{-1}x)=\phi(R_{-t} x).
\]
Dann ist $U(t)$ unitär und $U(t+s)=U(t) U(s)$. Zum Beweis der Stetigkeit genügt es zu zeigen, dass $U(t)$ auf $C_0^\infty (\R^3)=:D$ stark differenzierbar (und somit stark setig) ist. Es gilt $\overline{D}=H$ und $U(t)D\subset D$ für alle $t\in \R$. Wir zeigen nun noch dass $t\mapsto U(t)\phi$ differenzierbar für alle $\phi\in D$. Um einen Kandidaten für $\lim_{t\to 0}$ zu finden, differenzieren wir die Funktion $(U(t)\phi)(x)=\phi(R_{-t} x)$ partiell nach $t$. 

Für $\phi\in C_0^\infty(\R^3)$ gilt:
\begin{align*}
i \frac{\delta}{\delta t} \phi(R_{-t}x)|_{t=0} &= i \nabla \phi(x) \cdot \frac{d}{dt} R_{-t} x|_{t=0}\\
&=-i \nabla \phi(x) \cdot (e \land x) = (e\land x) \cdot (-i \nabla \phi(x))\\&= e\cdot (x \land(-i \nabla \phi(x)) =: (B\phi)(x).
\end{align*}
Wie im obigem Beispiel zeigt man  nun, dass 
\[
\|i \frac{U(t)\phi- \phi}{t} - B\phi\| \to 0
\]
mit Hilfe mit majorisierter Konvergenz. Aus dem Theorem von Nelson folgt, dass $\overline{B}$ der Erzeuger von $U$ ist. Formal ist
\[
B=\vec e \cdot \vec L, \quad \vec L=x \land (-i \nabla) =(L_1, L_2, L_3)
\]
Die drei Komponenten $L_1, L_2, L_3$ von $\vec L$ heißen \emph{Drehimpulsoperatoren}. Wir haben gezeigt: Der Abschluss von $L_k=\vec e_k \cdot \vec L \upharpoonright C_0^\infty$ erzeugt die "`Drehungen"' um die $k$-Achse. $\vec e_k$ ist dabei der $k$-ter Vektor der Standardbasis von $\R^3$.

\chapter{Nichtentartete Schrödingergleichungen}

\section{Existenz des Propagators}
Sei $I\subset \R$ ein Intervall und sei $A: I \mapsto \mathcal L(H)$ stark stetig mit $A(t)^*=A(t)$. Wir schreiben das AWP
\[
i \dot \phi = A(t) \phi, \phi(s)=\eta \in H
\]
als Integralgleichung:
\[
\phi(t)=\eta-i \int_\rho^t A(t_1) \phi(t_1) \dx[t_1]
\]
die wir iterieren. Im Konvergenzfall bekommen wir $\phi(t)=U(t,s) \eta$ wobei
\begin{align*}
U(t,s) \eta&= \eta- i \int_s^t A(t_1) \eta \dx[t_1]+ ... \\
&= \eta + \sum_{n=1}^\infty (-i)^n \int_s^t \dx[t_1] \int_s^{t_1} \dx[t_2] ... \int_s^{t_{n-1}} \dx[t_n] A(t_1)... A(t_n) \eta\\
&=T \exp(-i \int_s^t A(\tau) \dx[\tau])
\end{align*}
siehe physikalische Linteratur. $T$ entspricht dabei der Zeitordnung. Man bezeichnet die Reihe als \emph{Dysan-Reihe}.
\subsection{Dysan-Reihe}
Wir werden zeigen, dass $U(t,s)$ folgende Eigenschaften hat
\begin{enumerate}[a)]
\item $U(t,s)$ ist unitär.
\item $U(t, t)=1$ und $U(r,s)U(s,t)= U(r,t)$
\item $(t,s)\mapsto U(t,s) \eta$ stetig auf $J\times J$ für alle $\eta \in H$.
\end{enumerate}
Jede Familie $U(t,s)\in \mathcal L(H)$ mit den Eigenschaften (a), (b), (c) heißt unitärer \emph{Propagator} oder unitäres  \emph{Evolutionssystem}.

\begin{st}
Sei $A: J \to  \mathcal L(H), A(t)^*=A(t)$ und sei $t\mapsto A(t)$ stark stetig. Dann wird durch die Dysanreihe ein unitärer Propagator $U(t,s)$ definiert so, dass für alle $\phi \in H$ und $s,t \in I$
\begin{equation} \label{3.1}
i \frac{d}{dt} U(t,s)\phi = A(t) U(t,s) \phi, \quad i \frac{d}{ds} U(t,s) \phi = - U(t,s) A(s) \phi.
\end{equation}
Insbesonder löst $\phi(t)=U(t,s) \eta$ das AWP 
\[
i \dot \phi = A(t) \phi, \quad \phi_s=\eta.
\]
\end{st}
\begin{proof}
Sei $I \subset J$ ein kompaktes Teilintervall und $M= \sup_{\tau \in I} \|A(\tau)\|$. Aus $\sup_{\tau \in I} \|A(\tau) \phi\|<\infty$ und dem Prinzip über die gleichmäßige Beschränktheit 
\fixme

Also gilt für alle $s, t \in I, s<t$
\[
\int_s^t \dx[t_1] \int_s^{t_1} \dx[t_2] ... \int_s^{t_{n-1}} \dx[t_n] \|A(t_1) \cdots A(t_n) \eta\| \le M^n \frac{(t-s)^n}{n!} \|\eta\| \le \frac{(M|I|)^n}{n!} \| \eta\|
\]
wobei $|I|$ die Länge des Intervalls $I$ sei. Also ist die Dysanreihe absolut und gleichmäßig konvergenz für $s,t$ aus einem kompakten Teilintervall von $J$. Dasselbe gilt für die gliedweise nach $s$ oder $t$ differenzierte Dysanreihe. Daraus folgen  Eigenschaft (c) und \eqref{3.1} denn
\begin{align*}
&\frac{d}{ds} \int_s^t \dx[t_1] \int_{s}^{t_1} \dx[t_2] ... \int_s^{t_{n-1}} \dx[t_n]  A(t_1)...A(t_n) \eta \\ &=- \int_s^t \dx[t_1] \int_s^{t_1} \dx[t_2] ... \int_s^{t_{n-2}} \dx[t_{n-1}] A(t_1)... A(t_{n-1})A(s) \eta
\end{align*}
Nach \eqref{3.1} gilt
\[
i \frac{d}{dr} U(t,r) U(r,s)\eta= U(t,r)(-A(r)+A(r)) U(r,s))\eta=0.
\]
Also ist
\[
U(t,s)=U(t,s) \underbrace{U(s,s)}_{I}= U(t,r) U(r,s)
\]
für alle $r\in \R$. Also ist (b) gezeigt. Insbesondere gilt
\[
U(t,s) U(s,t)=U(t,t)=I=U(s,t) U(t,s).
\]
Also ist $U(t,s)$ bijektiv mit Inverse $U(s,t)$. Aus $A(t)^*=A(t)$ folgt:
\[
\frac{d}{dt} \|U(t,s) \eta\|^2=\frac{d}{dt} \< U(t,s) \eta, U(t,s) \eta \> = \< -iA(t) U(t,s)\eta, U(t,s) \eta\rangle + c.c.=0. 
\]
Hierbei bezeichne c.c. dem komplex konjugierten des vorigen. \fixme
\end{proof}

\begin{lem}\label{3.2}
Sei $U(t)$ eine s.s.u.G. mit Erzeuger $A$ und sei $t\mapsto \phi(t)\in H$ stetig auf $J\ni 0$. Dann ist auch
\[
\Phi(t)=\int_0^t U(t-s) \phi(s) \, \dx[s] = \int_0^t U(s) \phi(t-s) \, \dx[s]
\]
stetig. Falls $\phi$ stetig differenzierbar ist, dann ist  auch $\Phi$ stetig differnzierbar und $\phi(t) \in D(A)$ für alle $t\in \R$ und
\[
\dot \Phi(t)=U(t) \phi(0)+ \int_0^t U(t-s) \dot \phi(s) \, \dx[s]=\phi(t)-iA\underbrace{\int_0^t U(t-s) \phi(s) \, \dx[s]}_{\Phi(t)}
\]
\end{lem}
\begin{proof}
Es ist
\begin{equation}\label{3.2}
\Phi(t+\eps)-\Phi(t)=\int_0^t U(s) (\phi(t+\eps-s) - \phi(t-s))\, \dx[s] + \int_t^{t+\eps} U(s) \phi(t+\eps-s) \, \dx[s].
\end{equation}
Es folgt
\[
\| \Phi(t+\eps)- \Phi(t)\| \le \int_0^t \|\phi(t+\eps-s)- \phi(t-s) \| \dx[s] + \int_t^{t+\eps} \|\phi(t+\eps-s)\| \,\dx[s] \to 0
\]
mit $\eps \to 0$. Sei nun $\phi(t)$ stetig differenzierbar. Nach \eqref{3.2} folgt
\begin{align*}
\frac{1}{\eps} ( \phi(t+\eps)-\phi(t)) &= \int_0^t U(s) \frac{1}{\eps} ( \phi(t+\eps-s)-\phi(t-s))\,\dx[s]\\
&+ \frac{1}{\eps} \int_t^{t+\eps} U(s) \phi(t+\eps-s)\, \dx[s] \\ &\to \int_0^t U(s)\dot \phi(t-s) \dx[s] + U(t) \phi(0)
\end{align*}
für $\eps\to 0$. Dies ist stetig und hängt von $t$ ab nach dem zuvor bewiesenen. 
Eine zweite Darstellung von $\Phi$ ist gegeben durch
\[
\Phi(t+ \eps)-\Phi(t) = \int_0^t (U(t+\eps-s)-U(t-s)) \phi(s) \, \dx[s] + \int_t^{t+\eps} U(t+\eps-s) \phi(s) \, \dx[s].
\]
Und damit
\[
\frac{1}{\eps} (\Phi(t+\eps)-\Phi(t))= \frac{1}{\eps}(U(\eps)-1) \int_0^t U(t-s) \phi(s) \, \dx[s] + \frac{1}{\eps} \int_{t}^{t+\eps} U(t+\eps-s) \phi(s) \, \dx[s],
\]
wobei
\[
\frac{1}{\eps} \int_t^{t+\eps} U(t+\eps-s) \phi(s) \, \dx[s] \stackrel{\eps\to 0} \longrightarrow U(0) \phi(t) = \phi(t).
\]
Also ist $\phi(t) \in D(A)$ und
\[
\dot \phi(t)=\phi(t)-i A\phi(t).
\]
\end{proof}
Im Folgenden verwenden wir eine kleine Abwandlung dieses Lemmas
\begin{lem*}
$U$ ein s.s.U.G. mit Erzeuger $A$ und sei $\gamma: J \to H$ stetig. Dann ist auch
\[
\Phi(t,s)=\int_s^t U(t-r) \gamma(r) \, \dx[r]=\int_0^{t-s} U(r) \gamma(t-r) \,\dx[r]
\]
stetig auf $J\times J$. Falls $\gamma$ von der Klasse $C^1$ ist 
\end{lem*}
\begin{st}[R.S. Phillips 1953]
Sei $A_0=A_0^*$ Erzeuger einer unitären Gruppe $U_0$ und sei $t\to B(t)\in \mathcal L(H)$, $t\in J\subset \R$ \emph{stark stetig differenzierbar}. Dann existiert ein unitärer Propagator $U(t,s)$ mit
$U(t,s) D(A_0)\subset D(A_0)$ und wenn $\eta \in D(A_0)$ gilt
\[
i\frac{d}{dt} U(t,s)\eta= A(t) U(t, s) \eta
\]
mit $A(t)=A_0+B(t)$. Außerdem gilt
\begin{align*}
U(t,s)&=U_0(t-s)-i \int_s^t \dx[t_1] U_0(t-t_1) B(t_1) U(t_1-s) \eta \\ &+(-i)^2 \int_s^t \dx[t_1] \int_s^{t_1} \dx[t_2] U_0(t-t_1) B(t_1) U_0(t_1-t_2) B(t_2) U_0(t_2-s)\eta+ ...
\end{align*}
was sich für $A_0=0$ auf die bekannte Dysan-Reihe aus Satz \ref{3.1} reduziert.
\end{st}
\begin{nt*}
Man kann zeigen (siehe Phillips 1953) dass starke Stetigeit von $B(t)$ im Allgemeinen nicht hinreichend ist.
\end{nt*}
\begin{proof}
Wir merken an, dass $U_0(t,s)=U_0(t-s)$. Wir definieren
\[
U_1(t,s)\phi= U_0(t,s)\phi-i \int_s^t U_0(t,r) B(r) U_0(r,s) \phi\, \dx[r]
\]
und weiter
\[
U_n(t,s) \phi = U_0(t,s) \phi-i \int_s^t U_0(t,r) B(r) U_{n-1}(r,s) \phi \dx[r].
\]
Weiter definiere $W_0(t,s)=U_0(t-s)$  und 
\[
W_n(t,s)\eta= (U_n(t,s)-U_{n-1}(t,s))\eta= -i \int_s^t U_0(t,r) B(r) W_{n-1} (r,s) \eta \, \dx[s].
\] 
Nach Annahme über $B$ ist $r\mapsto B(r) U_0(r,s) \phi$ stetig. Also ist auch $t\mapsto U_1(t,s) \phi$ stetig nach Lemma \ref{3.2}.  Induktiv folgt mit Lemma \ref{3.2}, dass $t\mapsto U_n(t,s) \phi$ stetig ist für alle $n\in \N$. Insbesondere sind obige Integrale als Riemann-Integrale wohldefiniert. Wie im Beweis von Satz \ref{3.1} zeigt man nun, dass
\[
\|W_n(t,s)\| \le \frac{M^n}{n!} |I|^n,
\]
wobei $I\subset J$ kompaktes Intervall, $t,s\in I$ und $M=\sup_{t\in I} \| B(t) \| < \infty$. Also konvergiert die Reihe
\[
U(t,s)=\sum_{n=0}^\infty W_n(t,s)=\lim_{n\to \infty} U_n(t,s)
\]
in $\mathcal L(H)$ absolut und lokal bezüglich $t,s$ gleichmäßig. Da $U_n(t,s) \phi$ stetig in $t$ (und $s$) ist, ist auch $U(t,s)\phi$ stetig in $(t,s)\in J\times J$. Außerdem $U(t,t)=I$, denn $U_n(t,t)=I$.
Aus der Rekursionsbeziehung für $U_n(t,s)$ folgt im Grenzwert $n\to \infty$, dass
\[
U(t,s) \phi= U_0(t,s)\phi-i \int_s^t U_0(t,r) B(r) U(r,s) \phi\, \dx[r]
\]
Annahme: Für $\phi \in D(A)$ sei $t\mapsto U(t,s) \phi$ stetig differenzierbar. Dann ist auch $r\mapsto B(r) U(r,s) \phi$ stetig differenzierbar und aus Lemma \ref{3.2} folgt, dass
\[
\int_s^t U_0(t,r) B(r) U(r,s) \phi \, \dx[r] \in D(A_0)
\]
also auch $U(t,s)\phi \in D(A_0)$ und nach Lemma \ref{3.2}
\[
\frac{d}{dt} U(t,s) \phi=-i (A_0 +B(t)) U(t,s) \phi
\]
Es bleibt zu zeigen, dass $t\mapsto U(t,s)\phi$ in $C^1$ ist für $\phi\in D(A)$. Es gilt
\[
t \mapsto W_0(t,s) \phi = U_0(t,s) \phi
\]
ist stetig differenzierbar und
\[
\frac{\delta}{\delta t} W_0(t,s) \phi = U_0(t,s) (iA_0) \phi,
\]
was bezüglich $f$ stetig ist. Falls $t\mapsto W_{n-1} (t,s) \phi$ von der Klasse $C^1$, dan auch
\[
W_n(t,s) \phi=-i \int_s^t \dx[r] U_0(t,r) B(r) W_{n-1}(r,s) \phi
\]
nach Lemma \ref{3.2}, $W_n(t,s) \phi \in D(A_0)$ und
\begin{align*}
\dot W_n(t,s) \phi &= U_0(t,s) B(s) W_{n-1}(s, s) \phi\\
&- i\int_s^t U_0(r,t) \dot B(r) W_{n-1}(r,s) \phi \\
& -i\int_s^t U_0(t,r) B(r) \dot W_{n-1} (r,s) \phi \, \dx[r]. 
\end{align*}

Daraus folgt:
\[
\|\dot W_1(t,s)  \phi \| \le M \phi\| + |t-s| M \| \phi \| + |t-s| M \| A_0 \phi\|
\]
wobei $M:= \sup_{r\in I} \{ \|B(r) \| + \| \dot B(r) \| \} < \infty$. Für $n \ge 2$ gilt
\begin{align*}
\| \dot W_n(t,s) \phi \| &\le M \int_s^t( \|W_{n-1}(r,s) \phi \| + \| \dot W_{n-1} (r,s) \phi \|) \,\dx[r] \\ &\le M^n \frac{ |t-s|^n}{n!} \| \phi \| + M \int_s^t \| \dot W_{n-1}(r,s) \phi \| \dx[r].
\end{align*}
Mittels Induktion.
\begin{seg}[Induktionsannahme]
\[
\|\dot W_{n}(t,s) \phi \| \le  \frac{M^n}{(n-1)!}(|t-s|^n + |t-s|^{n-1}) \| \phi \|_A
\]
mit $\| \phi\|_A= \|\phi\|+ \|A \phi \|$. Die Annahme ist erfüllt für $n=1$ 
\end{seg}
\begin{seg}[Induktionsschritt]
Für $n\ge 2$ folgt
\begin{align*}
\| \dot W_n(t,s) \phi \| &\le \frac{M^n}{n!} |t-s|^n \| \phi \| \\
&+ M \int_s^t \frac{M^{n-1}}{(n-1)!} (|r-s|^{n-1} + |r-s|^{n-2}) \| \phi\|_A \, \dx[r]\\
&\le \big[M^n(t-s)^n (\underbrace{\frac{1}{n!} + \frac{1}{(n-2)! n}}_{=\frac{1}{(n-1)!}})+ M^n \frac{|t-s|^{n-1}}{(n-1)!} \big] \| \phi\|_A.  
\end{align*}
\end{seg}
Es folgt, dass die Reihe $\sum_{n} \dot W_n(t,s) \phi$ absolut und gleichmäßig konvergent ist für $t,s\in I$. Also gilt
\[
\dot U(t,s) \phi = \sum_{n\ge 0 } \dot W_n(t,s) \phi
\]
was bezüglich $t,s$ stetig. Also ist $t \mapsto U(t,s) \phi$ stetig differenzierbar für $\phi \in D(A)$. Dass $U(t,s)$ ein unitärer Propagator ist, ist eine Übungsaufgabe.
\end{proof}

\begin{nt*}
\fixme Mit letzteren Satz folgt
\begin{align*}
U(t,s) \phi- U_0(t,s) \phi &= U_0(t, r) U(r,s) \phi\big|_{r=s}^{r=t}\\
&= \int_{s}^t \frac{\delta}{\delta r} U_0(t,r) U(r,s) \phi \, \dx[r]\\&= \int_s^t U_0(t,r) (iA_0-i(A_0+B(r))) U(r,s) \phi \, \dx[r] \\ &= -i \int_s^t U_0(t,r) B(r) U(r,s) \phi \, \dx[r].
\end{align*}
Dies kann man iterieren und erhält
\begin{align*}
U(t,s) \phi &= U_0(t,s) \phi - i \int_s^t U_0(t,r) B(r) U_0(r,s) \phi \, \dx[r] \\ &+ (-i)^2 \int_s^t \dx[t_1] \int_s^{t_1} \dx[t_2] U_0(t, t_1) B(t_1) ....
= \sum_{n=0}^\infty W_n(t,s) \phi.
\end{align*}
\end{nt*}

\begin{st}[Kato 1953] \label{3.4}
Sei $A(t): D \subset H \to H,$ $t\in J$ eine Familie von selbstadjungierte Operatoren mit $D(A(t))=D$ für alle $t\in J$. Falls $t \mapsto A(t) \phi$ auf $J$ stetig differenzierbar ist für alle $\phi \in D$. Dann existiert ein unitärer Propagator $U: J \times J \to \mathcal L(H)$ mit
\begin{enumerate}[(i)]
\item $U(t,s) D\subset D$ für alle $t,s \in J$
\item Für $\phi\in D$
\[
\frac{d}{dt} U(t,s) \phi = - i A(t) U(t,s) \phi, \quad \frac{d}{ds} U(t,s) \phi= U(t,s) iA(s) \phi.
\]
\end{enumerate} 
\end{st}
\begin{proof}
s. Yosida: Functional Analysis, siehe auch Pazy, Engel und Nagel
\end{proof}
\begin{nt*}
Die zweite Gleichung von (ii) folgt aus der ersten
\begin{align*}
\frac{1}{\eps} ( U(t,s+\eps) \phi - U(t,s) \phi) &= U(t, s+\eps) \frac{1}{\eps} (\phi-U(s+\eps, s) \phi)\\
&= - U(t,s) \underbrace{\frac{1}{\eps} (U(s+\eps) \phi- \phi)}_{\to -iA(s) U(s,s) \phi}) \\ 
&\longrightarrow U(t,s) i A(s) \phi 
\end{align*}
für $\eps \to 0$.
\end{nt*}

\section{Das Adiabatische Theorem}
Sei $H(s), s\in [0,1]$ eine Familie von selbstadjungierten Operatoren in $H$ mit $D(H(s))=D$. Für jedes $s\in I$ sei
\[
\sigma(H(s)) = \sigma_1(s) \cup \sigma_2(s)
\]
wobei $\sigma_1(s) \cap \sigma_2(s) = \emptyset$ und
\[
\inf_{s\in I} \dist( \sigma_1(s), \sigma_2(s)) >0.
\]
Z. B. kann $\sigma_1(s)=\{E(s)\}$ wobei $E(s)$ diskreter Eigenwert von $H(s)$ ist. Sei
\[
P(s)=\frac{1}{2\pi i} \oint_\gamma (z-H(s))^{-1} \, \dx[z]
\]
mit $\gamma$ wie in der Figur:
\fixme[fig1]


Auch $\gamma$ hängt von $s$ ab.

Wir studieren die Lösung von
\begin{equation}\label{3.3}
i \dot \phi = H(\frac{t}{\tau}) \phi,\quad \phi(0)= P(0) \phi(0)
\end{equation}
für $t\in [0, \tau]$ im Grenzwert $\tau \to \infty$. Das Ziel ist zu zeigen, dass
\[
\sup_{t\in [0, \tau]} \| (1- P(t)) \phi(t) \| \to 0
\]
für $\tau \to \infty$. Dass heißt Überänge von $P(0) H$ nach $(1- P(t))H$ sind unmöglich (beziehungsweise sehr unwahrscheinlich) im Grenzwert $\tau \to \infty$. Wir definieren $s=\frac{t}{\tau}$, dann wird \eqref{3.3} äquivalent zu
\[
i \dot \phi(s) = \tau H(s) \phi(s), \quad \phi(0)= P(0) \phi(0)
\]
wobei $s\in [0,1]$. Wir nehmen an:
\begin{enumerate}[(i)]
\item $s\to H(s) \phi$ ist $C^1$ für alle $\phi\in D$
\item $s\mapsto P(s)$ ist von der Klasse $C^2(I)$.
\end{enumerate}
Annahme (ii) kann aus mehr Regularität von $s\mapsto H(s) \phi$ hergeleitet werden. Annahme (i) garantiert nach dem Satz von Kato \ref{3.4} die Existenz des Propagators $U(t,s)$ mit
\[
i \frac{d}{dt} U(t,s) \phi= \tau H(s) U(t,s) \phi, \quad U(t,t) = I.
\]
Wir definieren eine \emph{adiabatische} Evolution $U_A(t,s)$ durch die Schrödingergleichung zu $\tau H_A(s)$ wobei
\[
H_A(s):= H(s) + \frac{i}{\tau} [\dot P(s), P(s)].
\]
wobei $[\dot P, P]:= \dot P P - P \dot P$ den Kommutator definiert.
\begin{lem} Für alle $t,s \in I, \tau >0$ ist
\[
U_A(t,s) P(s) = P(t) U_A(t,s).
\]
\end{lem}
\begin{proof}
Aus $P=P^2$ folgt $\dot P=\dot P + P \dot P$ und somit auch
\[
P\dot P = P \dot P P + P \dot P.
\]
Und damit $P\dot P P =0$ (d.h. $\dot P P = (1-P) \dot PP$). Es folgt, im \emph{schwachen} Sinn,
\begin{align*}
\frac{d}{dt} U_A(s,t) P(t) U_A(t,s)&= U_A(s,t) (i \tau H_A(t) P+ \dot P(t)-P i\tau H_A(t))U_A(t,s)\\
&=U(s,t)(-[\dot P, P] P+ \dot P + P [\dot P, P])_t U_A(t,s)\\
&= U_A(s,t) (\underbrace{-\dot P P + \dot P - P \dot P}_{-(\dot P P + P \dot P) + \dot P=0}) U_A(t,s)=0.
\end{align*}
Es folgt
\[
U_A(s,t) P(t) U_A(t,s) = U_A(s,t) P(t) U_A(t,s)\big|_{t=s} = P(s).
\]
Und es folgt $U_A(s,t)P(t)=P(s) U_A(s,t)$.
\end{proof}

\begin{lem}\label{3.6}
Ist $J\subset \R$ ein Intervall und $B: J \mapsto \mathcal L(H)$ stark stetig differenzierbar, dann ist $s\mapsto B(s)$ norm-stetig. (Allgemeiner gilt: Ist $t\mapsto B(s) \phi$ von der Klasse $C^{k+1}$ für alle $\phi \in H$, dann ist $s\mapsto B(s)$ von der Klasse $C^k$ bezüglich der Operatornorm).
\end{lem}

\begin{proof}
Für alle $\phi\in H$ gilt:
\[
B(t) \phi - B(s) \phi = \int_s^t \dot B(r) \phi\, \dx[r]
\]
wobei $\dot B(r) \phi = \frac{d}{dr} B(r) \phi$ stetig bezüglich $r$ ist. Also gilt ($s\in J$ fest)
\[
\sup_{|r-s|\le \eps} \| \dot B(r) \| < \infty
\]
und somit $M_\eps:= \sup\|_{|r-s|\le \eps} \dot B(r)\| < \infty$ nach dem Prinzip über gleichmäßige Beschränktheit. Also
\[
\| B(t) \phi - B(s) \phi \| \le \big | \int_s^t \overbrace{\| \dot B(r) \phi\|}^{\le M_\eps \| \phi\|} \, \dx[r]\big | \le | t-s| M_\eps\| \phi\|
\]
für $|t-s|\le \eps$. D.h.
\[
\| B(t)-B(s)\| \le |t-s| M_\eps \to 0
\]
für $t\to s$.
\end{proof}


\begin{lem}\label{3.7}
Sei $s\mapsto H(s) \phi$ stetig differenzierbar für alle $\phi \in D$. (Ann. (i)) und sei $z\in \C$. Dann ist $\{s\in I | z\in \rho(H(s))\}$ offen, $s\mapsto (z-H(s))^{-1}$ ist stark stetig differenzierbar und
\[
\frac{d}{ds} (z-H(s))^{-1}\phi = (z_H(s))^{-1} \dot H(s) (z- H(s))^{-1} \phi.
\]
\end{lem}
\begin{proof}
Sei $s\in I$ fest mit $z\in \rho(H(s))$. Dann gilt für $t\in I$
\begin{align*}
(z-H(t)) &= (z- H(s) - (H(t)-H(s))\\
&= (1-(H(t)-H(s)) R(s))(z-H(s))
\end{align*}
wobei $t\mapsto H(t) R(s)$ stark stetig differenzierbar und somit, nach Lemma \ref{3.6}, norm-stetig ist. D.h.
\[
\|(H(t)-H(s)) R(s)\|= \|H(t) R(s)- H(s) R(s) \| <1
\]
für $|t-s|$ klein genug. Für solche $t$ ist $z\in \rho(H(t))$ und
\[
(z- H(t))^{-1} = (z-H(s))^{-1} (1-\underbrace{(H(t)-H(s)) R(s)}_{=B(t)})^{-1}
\]
Da die Abbildung $B\mapsto B^{-1}$ stetig ist auf der offenen Menge der invertierbaren Elemente von $\mathcal L(H)$, folgt dass $t\mapsto (z-H(t))^{-1}$ norm-stetig ist. Somit gilt für alle $\phi \in H$
\begin{align*}
\frac{1}{h} [(z-H(s+h))^{-1} &- (z-H(s))^{-1}]\phi\\
&=(z-H(s+h))^{-1} [ \frac{H(s+h)-H(s)}{h}] (z-H(s))^{-1}\phi \\
&\to (z-H(s))^{-1} \dot H(s) (z-H(s))^{-1} \phi
\end{align*}
mit $h\to 0$, was stetig von $s$ abhängt. (Übung)
\end{proof}
\begin{lem}\label{3.8}
Für $s\in I$ sei
\[
X(s):= - \frac{1}{2\pi i} \oint_\gamma(z-H(s))^{-1} \dot P(s) (z-H(s))^{-1} \, \dx[z]
\]
wobei $\gamma$ wie in der Definition von $P$ gewählt wird. Dann gilt
\begin{enumerate}[(i)]
\item $[H,X]=[\dot P, P]$ auf $D$
\item $s\mapsto \chi(s)$ ist stark stetig differenzierbar.
\end{enumerate}
\end{lem}
\begin{proof}
Im schwachen Sinn, d.h. für Skalarprodukte mit Vektoren $\psi, \phi \in D$:
\begin{align*}
[H,X] &= - \frac{1}{2\pi i} \oint_\gamma [H, (z-H)^{-1} \dot P(z-H)^{-1} ]\,\dx[z]\\ 
&= \frac{1}{2\pi i} \oint_\gamma [ z-H, (z-H)^{-1} \dot P (z-H)^{-1} ] \, \dx[z]\\
&= \frac{1}{2\pi i} \oint_\gamma (\dot P(z-H)^{-1} - (z-H)^{-1} - (z-H)^{-1} \dot P)\, \dx[z] \\
&= \dot P P - P\dot P= [\dot P, P]
\end{align*}
Und es folgt (i). (ii) folgt aus $P\in C^2(I, \mathcal L))$ und Lemma \ref{3.7}.
\end{proof}
\begin{st}[Adiabatische Theorem] \label{3.9}
Sei $U(s)=U(s,0)$, $U_A(s) = U_A(s,0)$. Dann gilt
\[
\sup_{s\in I} \| U(s) - U_A(s)\| = \mathcal O(\frac{1}{\tau})
\]
für $\tau \to \infty$.
\end{st}
\begin{proof}
$\| U(s)-U_A(s)\| = \|1- U(s)^* U_A(s)\|$ wobei für jedes $\phi \in D$,
\begin{align*}
U(s)^* U_A(s) \phi - \phi &= \int_0^s U(t)^*(i\tau H(t) - i \tau H_A(t)) U_A(t) \phi \, \dx[t]\\
 &= \int_0^s U(t)^* [\dot P, P] U_A(t) \phi \, \dx[t]\\
 &= \int_0^s U(t)^* [H,X] U(t)) U(t)^* U_A(t) \phi \dx[t] 
\end{align*}
wobei 
\begin{align*}
\frac{d}{dt} U(t)^* X(t) U(t) &= U(t)^* [i \tau H, X] U(t) + U^* (t) \dot X U(t) \\
&= U(t)^*[i \tau H, X] U(t) + U^*(t) \dot X U(t).
\end{align*}
Es ist
\[
U^*[H,X] U= \frac{1}{i \tau} \frac{d}{dt} U^* X U - \frac{1}{i \tau} U^* \dot X U.
\]
Also 
\begin{align*}
&U^*(s) U_A(s) \phi - \phi= \frac{1}{i \tau} \int_0^s \frac{d}{dt} ( U^* X U) U^* U_A \phi \, \dx[t] - \frac{1}{i \tau} \int_0^s U^* \dot X U \, \dx[t] \\
&\stackrel{\text{part. Int.}}= \frac{1}{i \tau} U^* X U_A(s) \phi - \frac{1}{i \tau} \int_0^s U^* X [\dot P, P] U_A \phi \, \dx[t]- \frac{1}{i\tau} \int_0^s U^* \dot X U \phi\, \dx[t]
\end{align*}
Daraus folgt

\[
\| U(s) - U_A(s) \| \le \frac{1}{ \tau} \sup_{s\in I} ( \| X(s)\| + \| X [\dot P, P] \| + \| \dot X\| )
\]
\end{proof}
\begin{kor}\label{3.10}
Es gilt
\[
\sup_{s\in I}\| (1-P(s)) U(s) P(0)\| = \mathcal O(\frac{1}{\tau})
\]
für $\tau \to \infty$.
\end{kor} 
\begin{proof}
Es folgt
\begin{align*}
\|(1-P(s))U(s) P(0)\| &\stackrel{Lem. \ref{8.5}}= \|(1-P(s))(U(s)-U_A(s))P(0)\|\\ &\le \|U(s)-U_A(s)\| = \mathcal O(\frac{1}{\tau})
\end{align*}
nach Theorem \ref{3.9}.
\end{proof}

Es gilt $U_A(t,s) P(s)=P(t) U_A(T,s)$. Wir betrachten Schrödingergleichung der Form
\[
i \dot \phi = H(t/ \tau) \phi, \quad s=t/\tau, \quad \frac{d}{ds} = \tau \frac{d}{dt}
\]
mit $\tilde \phi = \phi(t) = \phi(\tau s)$ und erhalten die Modifikation
\[
i \frac{d}{ds} \tilde \phi = \tau \dot \phi (\tau s) = \tau H(s)  \phi(\tau s) = \tau H(s) \tilde \phi(s)
\]

\begin{seg}[Annahm]
für alle $\phi \in D$ ist $H(s) \phi$ stetig differenzierbar in S, dann gibt es ein $U(t_1, t_2)$.
Und ein großes \fixme[fig2].
\end{seg}

\subsection{Adiabatisches Theorem ohne spektrale Lücke}

\begin{seg}[Annahmen]
$I=[0,1]$
\begin{enumerate}[i]
\item $s\mapsto H(s) \phi$ ist in $C^1(I)$ für alle $\phi \in D$.
\item Es gibt eine Abbildung $P: I \to \mathcal L(H)$ ist von der Klasse $C^2(I)$ mit $P^2= P= P^*$ und $P(S) H \subset D$ und $H(s) P(s)= E(s) P(s)$ wobei $E(s) \in \R$. Für fast alle $s\in I$ sei $P(s)$ der Projektor auf den Eigenraum von $H(s)$ zu $E(s)$.
\item $\dim P(s) H < \infty$.
\end{enumerate}
\end{seg}

\begin{thm}[Bornemann/Avrau und Elgert 1998]
Unter obigen Annahmen gilt
\[
\sup_{s\in I} \| U(s) - U_A(s) \| \to 0
\]
für $\tau \to \infty$.
\end{thm}
\begin{proof}
Wie im Beweis von Theorem 9 gilt:
\begin{align*}
U(s)^* U_A(s) \phi - \phi &= \int_0^s U^*[\dot P, P] U_A \phi \, \dx[r]\\
&= \int_0^s U^*([H, X_\eps] - Y_\eps) U_A\phi \dx[r]\\
&= \int_0^s U^*[H, X_\eps] U_A \phi \dx[r] - \int_0^s U^* Y_\eps U_A \phi \dx[r]\\
&\stack{\text{vgl. Thm } \ref{3.9}}= \frac{1}{i \tau} U^* X_\eps U_A \phi \big |_0^s - \frac{1}{i\tau} \int_0^s U^* X_\eps [\dot P, P] U_A \phi \, \dx[r]\\
&- \frac{1}{i\tau} \int_0^s U^* \dot X U_A \phi \, \dx[r] - \int_0^s U^* X_\eps U_A \phi \, \dx[r].
\end{align*}
wobei $\eps >0$ und $X_\eps$ wie in Blatt 7 gewählt. Also ist
\begin{align*}
\sup_{s\in I} \| U(s) - U_A(s)\| &\le \frac{1}{\tau} \sup_s( 2 \| X_\eps(s)\| + \| X_\eps \dot P P\| \\
&+ \int_0^1 \dx[r] ( \| \dot X_\eps \|/\tau + \| Y_\eps\|).
\end{align*}
Wir erinnern, dass $P\dot P P=0$, wobei
$\int_0^1\| Y_\eps\|\, \dx[r]$ für $\eps\to 0$ da $\| Y_\eps(r)\| \to 0$ für fast alle $r\in [0,1]$ (vgl. Blatt 7). Also wähle erst $\eps>0$ klein, dann für gegeben $\eps$ fest $\tau$ groß. Dann folgt die Behauptung.
\end{proof}

\subsection{Anwendungen}

\begin{enumerate}[1)]
\item \fixme[fig4]
$P(s)$ sei spektraler Projektor zu $E(s)$ für $s\neq s_0$ und $E(s)$ sei diskreter Eigenwert.
\item Eingebetteter Eigenwert \fixme[fig4]
$E(s)=\inf \sigma(H(s))$ ein Eigenwert endlicher Vielfachheit und sei $P(s)$ Eigenprojektor zu $E(s)$.
\end{enumerate}

\chapter{Schrödingeroperatoren}

\section{Multiplikations- und Differentialoperatoren}
Sei $( \Omega, \mu)$ ein $\sigma$-endlicher Maßraum (z.B. $\Omega = \R^n$, $\mu$ das Lebesgue-Maß). Sei $f: \Omega \to \C$ messbar und sei
\[
M_f : D(M_f) \subset L^2( \Omega) \to L^2( \Omega)
\]
definiert durch $M_f \phi:= f\phi$ wobei $D(M_f)=\{ \phi \in L^2| f\phi \in L^2(\Omega)\}.$

\begin{st}\label{4.1}
\begin{enumerate}[a)]
\item $M_f$ ist dicht definiert und $M_{\bar f} = M_f^*$
\item Das Spektrum von $M_f$ ist der wesentliche Wertebereich von $f$, d.h.
\[
\sigma(M_f) =\{z\in \C| \forall \eps >0, \mu(f^{-1} (B_\eps (z))>0\}
\]
\end{enumerate}
\begin{nt*}
Falls $f\in L^\infty(\Omega)$, dann 
\[
D(M_f) = L^2(\Omega) \text{ und } \| M_f\| \le \| f\|_\infty.
\]
\end{nt*}
\end{st}
\begin{proof}
siehe Satz 3.1 im Vorlesungsskript.
\end{proof}

\section{Differentialoperatoren}
Sei $\mathcal F. \mathcal L^2(\R^n) \to L^2(\R^n)$ die Fouriertransformation und sei $f: \R^n \to \C$ messbar. Dann gilt
\[
f(-i \nabla) = \mathcal F^{-1} M_f \mathcal F.
\]
Dabei sei 
\begin{align*}
D(-i \nabla)&=\{\phi \in L^2| \hat \phi \in D(M_f)\}\\
&=\{ \phi \in L^2| f \hat \phi \in L^2( \Omega)\}.
\end{align*}
Dann ist nach Satz \ref{4.1}, $f(-i \nabla)$ dicht definiert, abgeschlossen und $\sigma(f(-\nabla))=\sigma(M_f)$.

Der \emph{Laplace-Operator} $\Delta$ definiert durch
\[
- \Delta = \mathcal F^{-1} M_f \mathcal F, \quad f(p)=p^2
\]
ist selbstadjungiert auf
\[
D(-\Delta)=\{\phi \in L^2| \int| p^2 \hat \phi (p)|^2 \, \dx[p] < \infty\}=H^2(\R^n).
\]
und $\sigma(- \Delta) = [0, \infty)$. Die \emph{Resolvente} $R(z)=(z+\Delta)^{-1}$ durch
\[
R(z)=\mathcal F^{-1} \frac{1}{z-p^2} \mathcal F.
\]    
Für $n\le 3$ ist $\hat G_z(p):= (z-p^2)^{-1}$ quadrat-integrierbar. Somit ist $G_z:=F^{-1} \hat G_z \in L^2(\R^n)$ und es gilt (Übung)
\[
(R(z) \phi)(x)=\int G_z(x-y) \phi(y) \, \dx[y]
\]
Für $n=3$ und $z\in \C\setminus [0,\infty)$ ist
\[
G_z(x) = \frac{1}{4\pi |x|} e^{-\sqrt{-z}|x|}, \quad \Re \sqrt{-z} >0.
\]
Die \emph{unitäre Gruppe} $U(t)$ welche durch $- \Delta$ erzeugt wird ist gegeben durch
\[
U(t)=\mathcal F^{-1} e^{ip^2 t} \mathcal F.
\]
Für $\psi \in L^1\cap L^2(\R^n)$ und $\pm t >0$ gilt
\[
(U(t)\psi)(x)=\frac{e^{\mp i \pi}}{(4\pi |t|)^{n/2}} \int e^{i(x-y)^2/(4t)} \psi(y) \, \dx[y]
\]

\section{Sobolevräume}
Für $s\ge 0$ definieren wir
\[
H^s(\R^n) = \{u \in L^2| |p|^s \hat u(p) \text{ ist quadratintegrierbar}\}
\]
und für $u,v\in H^s(\R^n)$
\[
\langle u, v \rangle_S = \in \overline{\hat u(p)} \hat v(p) (1+p^2)^s \, \dx[p].
\]
Nach Plancherel ist $H^0(\R^n)=L^2(\R^n)$ und
\[
s>t \ge 0 \implies H^s(\R^n) \subset H^t(\R^n) \subset L^2(\R^n).
\]
\begin{lem}\label{4.2}
Sei $\hat u: \R^n \to \C$ eine gegebene $L^1$-Funktion und sei $k\in \N$. Falls $|p|^k \hat U(p)$ integrierbar ist, dann ist
\[
u(x)=\int e^{ipx} \hat u(p) \, \dx[p]
\]
stetig und $k$ Mal stetig differenzierbar und
\[
\delta^\alpha u(x) = \int e^{ipx} (ip)^\alpha \hat u (p) \, \dx[p]
\]
für $|\alpha| \le k$ mit$\alpha=(\alpha_1,..., \alpha_n) \in \N_0^n$.
\end{lem}
\begin{proof}
Übung: Verwende Riemann-Lebesgue, Lebesgue Majorisierte Konvergenz und $|e^{is} -1|\le |s|$.
\end{proof}

\begin{st}[Sobolev-Lemma]\label{4.3}
Sei $u\in H^s(\R^n)$ und $k\in \N_0$ mit $k<s- \frac{n}{2}$. Dann gilt für alle $\alpha: |\alpha|\le k$:
\begin{enumerate}[a)]
\item $u\in C^k(\R^n)$ und $\delta^\alpha u(x) \to 0$ für $|x|\to \infty$;
\item $\sup_x |\delta^\alpha u(x)| \le C_{k,s,n} \|u\|_s.$
\end{enumerate}
\end{st}
\begin{proof}
Übung: Zeige, dass $|p|^k \hat u(p)$ integrierbar ist und verwende Lemma \ref{4.2}. Siehe auch Satz 3.3 aus dem Vorlesungsskript.
\end{proof}

\begin{ex*}
Wir wissen, dass $H^2(\R^2)=D(-\Delta)$ und nach Satz \ref{4.3} gilt:
\begin{itemize}
\item $\phi \in H^2(\R^3) \implies \phi \in C(\R^3)$ und $\phi(x)\to 0$ ($|x|\to \infty$)
\end{itemize}
\fixme
\end{ex*}


\begin{st} 
$H^s(\R^n)$ ist ein Hilbertraum und $C_0^\infty$ ist dicht in $H^s(\R^n)$.
\end{st}
\begin{proof}
Übung, oder Literatur (z.B. Rudin).
\end{proof}
\begin{df}
Sind $u,v\in L^1_{\text{loc}}(\R^n)$ und $\alpha \in \N_0^n$, dann heißt $v$ \emph{schwache $\alpha$-Ableitung} von $u$, wir schreiben $v=\delta^\alpha u$, wenn für alle $\phi \in C_0^\infty (\R^n)$ gilt
\[
(-1)^{|\alpha|} \int \delta^\alpha \phi(x) u(x) \, \dx = \int \phi(x) v(x) \, \dx.
\] 
Die schwache $\alpha$-Ableitung ist eindeutig, denn aus $\int \phi v \, \dx=0$ für alle $\phi \in C_0^\infty$ folgt $v(x)=0$ fast überall. (Fundamentallemma der Distributionstheorie).
\end{df}

\begin{st}\label{4.5}
Sei $\alpha\in \N_0^n$ und $u\in L^2(\R^n)$. Dann sind äquivalent:
\begin{enumerate}[a)]
\item$u$ hat eine schwache $\alpha$-Ableitung $\delta^\alpha u\in L^2$.
\item $p \mapsto p^\alpha \hat u(p)$ ist quadratintegrierbar.
\end{enumerate}
Gilt a) oder b) dann $\delta^\alpha u = \mathcal F^{-1} (ip)^\alpha \mathcal Fu$.
\end{st}
\begin{proof}
Sei $D^\alpha$ der s.a. Operator definiert durch
\[
D^\alpha = \mathcal F^{-1} p^\alpha \mathcal F.
\]
Hierbei sei 
\[
\mathcal D(D^\alpha) = \{ u \in L^2| p^{\alpha} \hat u(p) \text{ ist in } L^2\}
\]
der Definitionsbereich von $D^\alpha$. 
\fixme

was äquivalent ist zu (a). Umgekehrt folgt aus (a), dass für alle $\phi \in C_0^\infty(\R^n)$ gilt:
\[
(-1)^{|\alpha|} \int \overline{\delta^\alpha \phi(x)} u(x) \, \dx = \int \overline{\phi(x)} \delta^\alpha u(x) \, \dx
\]
was äquivalent ist zu
\[
\langle D^\alpha \phi, U \rangle = \langle \phi, \delta^\alpha u \rangle (-i)^{|\alpha|}
\]
für alle $\phi \in C_0^\infty(\R^n)$. Daraus folgt
$\big(D^\alpha\upharpoonright C_0^\infty(\R^n)\big)^*u= (-i)^{|\alpha|} \delta^\alpha u$. Es genügt zu zeigen, dass
\[
D^\alpha = \overline{D^\alpha \upharpoonright C_0^\infty(\R^n)},
\]
d.h. dass $D^\alpha \upharpoonright C_0^\infty(\R^n)$ wesentlich s.a. ist. (Übung)
\end{proof}

\begin{st}\label{4.6}
Sei $u\in L^2(\R^n)$, $m\in \N$ und für alle $\alpha\in \N_0^n, |\alpha|\le m$ existiere die schwache $\alpha$-Ableitung $\delta^\alpha u$ und $\delta^\alpha u\in L^2$. Dann ist $u\in H^m(\R^n)$.  
\end{st}
\begin{proof}
Nach Satz \ref{4.5} ist $p^\alpha \hat u(p)$ quadratintegrierbar für alle $|\alpha|\le m$ und smit $|p|^m \hat u(p)$ quadratintegrierbar, also $u\in H^m$.
\end{proof}

\begin{kor}\label{4.7}
$C_0^m(\R^n) \subset H^m(\R^n)$
\end{kor}

\begin{st}\label{4.8}
Sei $u\in H^m(\R^n)$, $f\in C^m(\R^n)$ wobei alle Ableitungen $\delta^\alpha f, |\alpha| \le m$, in $L^\infty(\R^n)$ sind. Dann gilt $fu \in H^m$,
$\|fu\|_m\le C_f \| u\|_m$ und es gilt die Leibnizformel
\[
\delta^\alpha(fu)=\sum_{\beta \le \alpha} \binom{\alpha}{\beta} \delta^\beta f \delta^{\alpha- \beta} u.
\]
\end{st}
\begin{proof}
Wir setzen $\| u \|_{m,2}=\sum_{|\alpha| \le m} \| \delta^\alpha u\|_{L^2}$.
\end{proof}
Diese Norm ist äquivalent zur Norm von $H^m$. Für $\phi \in C_0^\infty(\R^n)$ gilt
\begin{equation}\label{4.1}
\delta^\alpha(f\phi) = \sum_{\beta \le \alpha} \binom{\alpha}{\beta} \delta^\beta f \delta^{\alpha-\beta} \phi
\end{equation}
und somit
\begin{align*}
\|\delta^\alpha(f\phi) \|_2 &\le \sum_{\beta \le \alpha} \binom{\alpha}{\beta} \| \delta^\beta f\| _{L^2} \| \delta^{\alpha-\beta} \phi \|_{L^2}\\
&\le C_{\alpha,f} \sum_{|\alpha|\le m} \| \delta^\alpha \phi \|_{L^2}= C_{\alpha,f} \| \phi \|_{m,2}.
\end{align*}
Daraus folgt
\[
\|f\phi\|_{m,2}= \sum_{|\alpha|\le m} \| \delta^\alpha(f\phi)\|_2=\underbrace{\big(\sum_{|\alpha|\le m} C_\alpha,f\big) }_{C_f}  \|\phi\|_{m,2}
\]
Somit ist $M_f:C_0^\infty (\R^n) \subset H^m \to H^m$ eine beschränkte lineare Abbildung. Sie hat eindeutige beschränkte Fortsetzung auf $H^m$, welche ebenfalls durch  Multiplikation mit $f$ gegeben ist. (Details dazu in Skript) Die Leibniz-Formel auf $H^m$ folgt nun aus \eqref{4.1} durch ein Approximationsargument.







\appendix
\chapter{Appendix}
\section{Beschränkte Operatoren}
Ein (linearer) Operator $A:H\to H$ heißt \emph{beschränkt}, falls es ein $C\ge 0$ gibt mit $\|Ax\| \le C\|x\|$ für alle $x\in H$. Dazu sind äquivalent
\begin{enumerate}[a)]
\item $A$ ist beschränkt,
\item $A$ ist stetig,
\item $A$ ist stetig im Punkt $x=0$.
\end{enumerate} 
Wir definieren eine Norm in $\mathcal L(H)=\{A: H\to H| A \text{ linear und beschränkt}\}$ durch
\[
\|A\|=\sup_{x\in H, \|x\|=1} \|Ax\|= \sup_{\|x\|\le 1} \|Ax\|=\sup_{x,y\in H, \|x\|=\|y\|=1} |\langle y, Ax\rangle\|.
\]
Damit wird $\mathcal L(H)$ zum Banachraum.
\begin{st}[Neumannreihe]
Sei $A\in \mathcal L(H)$ und 
$\sum_{n=0}^\infty A^n$ sei konvergent (z.B. für $\|A\|<1$). Dann ist $I-A: H\to H$ bijektiv und
\[
(I-A)^{-1}=\sum_{n=0}^\infty A^n=I+A+A^2+...
\]
\end{st}
\begin{st}
Sei $A\in \mathcal L(H)$, dann ist $\sigma(A)\subset \{z\in \C |\, |z|\le \|A\| \}$ und für $|z| < \|A\|$ gilt
\begin{equation}
(z-A)^{-1}=\sum_{n=0}^\infty \frac{A^n}{z^{n+1}} \label{A.1}
\end{equation}
\end{st}
\begin{nt*}
\eqref{A.1} gilt für alle $z$  mit
\[
|z| > \lim_{n\to \infty} \|A^n\|^{1/n}=\sup_{w\in \sigma(A)} |w|.
\]
Wir bezeichnen mit $r(A):= \sup_{w\in \sigma(A)} |w|$ den \emph{Spektralradius} von $A$.
\end{nt*}
\begin{proof}
Wendet man das Wurzelkriterium auf die Reihe $\sum \frac{A^n}{z^{n+1}}$ an, so ergibt sich
\[
\limsup_{n\to \infty} \big \|\frac{A^n}{z^n} \big \|^{1/n} <1 \iff \limsup_{n\to \infty} \|A^n\|^{1/n}< |z|.
\]
Man kann zudem zeigen, dass
\[
\limsup\|A^n\|^{1/n} = \lim_{n\to \infty} \|A^n\|^{1/n}=\sup_{w\in \sigma(A)} |w|.
\]
\end{proof}
\begin{st} \label{A.3}
Sei $A:D\subset H\to H$ dicht definiert $(\bar D=H)$ und sei $A$ beschränkt. Dann existiert genau ein beschränkter Operator $B\in \mathcal L(H)$ mit $Ax=Bx$ für alle $x\in D$. Es gilt $\|B\|=\|A\|$
\end{st}
\begin{proof}[Beweisidee:]
Sei $x\in H$ und $(x_n)$ eine Folge in $D$ mit $x_n \to x$. Dann ist $(Ax_n)$ eine Cauchy-Folge, also existiert $Bx:= \lim_{n\to \infty} Ax_n$. $B$ ist wohldefiniert und linear.
\end{proof}
\begin{ex*}
Betrachte die Fouriertransformation $\mathcal F: \mathcal S(\R^n)\to S(\R^n)$, wobei $\mathcal S(\R^n)$ den Schwartzraum bezeichne. Dann gilt
\[
\mathcal F\phi(p)=(2\pi)^{-n/2} \int e^{-ipx} \phi(x) \dx.
\]
Dann ist $\mathcal F \phi\in \mathcal S(\R^n)\subset L^2(\R^n)$ und $\|\mathcal F \phi\|=\|\phi\|$ für alle $\phi \in S(\R^n)$. 

Nach Satz \ref{A.3} existiert genau eine beschränkte Fortsetzung $\mathcal F: L^2(\R^n) \to L^2(\R^n)$.
\end{ex*}



\end{document}